\documentclass[11pt, a4paper]{article}

%--- Packages -------------------------------------------------------
\usepackage[english]{babel}
\usepackage[T1]{fontenc}
\usepackage[utf8]{inputenc}
\usepackage{fullpage}
\usepackage{amsmath}
\usepackage{amsfonts}
\usepackage{amssymb} 
\usepackage{graphicx} 
\usepackage[labelsep=colon,font={it},labelfont=bf,figureposition=below,tableposition=above]{caption} 
\usepackage{url}
\usepackage{siunitx}
\sisetup{exponent-product = \cdot, output-product = \cdot}
\usepackage{physics}
\usepackage{hyperref}

%--------------------------------------------------------------------

%--- Title Information ----------------------------------------------
\title{\textbf{S-Couppled Oscillations} \\
	\normalsize{Date of the experiment: October 23, 2024}} %Fill in the name of the experiment and the day it was performed
\date{\today}  %Date the report was written
\author{\normalsize{Kubik, David - email address} \\ %your name and email address
	    \normalsize{Assistant: NAME}} % Name of the assistant
%--------------------------------------------------------------------

%--- Your lab report ------------------------------------------------
\begin{document}

\maketitle


\section{Introduction}

Goal of this Lab was to measure some special cases of coupled oscillation both qualitatively and quantitatively. And compare these measurements to the the expectation derived in the Lab Instructions. In order to compare the compare the relationship between oscillation and spring constant $k$ . The constants of the springs used in the experiment were measured.
\begin{equation}
\omega_1 = \sqrt{\frac{k}{m}}
\end{equation}
\begin{equation}
\omega_2 = \sqrt{\frac{k + 2k'}{m}}
\end{equation}
\begin{equation}
\Omega = \omega_2 - \omega_1
\end{equation}

\begin{equation}
\omega' = \sqrt{\frac{k + k'}{m}}
\end{equation}


\section{Experimental setup} \label{sec:ex}

\subsection{Measurement of the Oscillation Period $T$}
As friction forces are completely ignored in the derivations, the experiment is built
to minimise friction forces. The oscillators consist of two riders, that glide on an air cushion and
that were coupled with each other using springs.


\subsection{Measurement of the Spring constants $k$}
\subsubsection{Measurement of $k$ using $\Delta h$}
\subsubsection{Measurement of $k$ using the Oscillation Period $T$}




\section{Measurement protocol} %how we acquire data and analyze it
\begin{center}
\begin{tabular}{ccc}
\toprule
$T_1$ (in s) & $T_2$ (in s) & T' (in s) \\
\midrule
1 & 1 & 1 \\
1 & 1 & 1 \\
1 & 1 & 1 \\
1 & 1 & 1 \\
1 & 1 & 1 \\
1 & 1 & 1 \\
1 & 1 & 1 \\
1 & 1 & 1 \\
1 & 1 & 1 \\
1 & 1 & 1 \\
1 & 1 & 1 \\
1 & 1 & 1 \\
1 & 1 & 1 \\
1 & 1 & 1 \\
1 & 1 & 1 \\
1 & 1 & 1 \\
1 & 1 & 1 \\
1 & 1 & 1 \\
1 & 1 & 1 \\
1 & 1 & 1 \\
\bottomrule
\end{tabular}

\end{center}






\section{Discussion} %Analysis






\section{Conclusion}



\begin{thebibliography}{9}%This section is optional if you decide to put your references directly in the text.




\bibitem{nist}
  \emph{The NIST Reference on Constants, Units, and Uncertainty},
  NIST,
  2017. \\
  \url{https://physics.nist.gov/cgi-bin/cuu/Value?r}
  Accessed Sept 2018
\end{thebibliography}

\end{document}
